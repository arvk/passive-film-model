\documentclass{article}

\usepackage{amsmath}
\usepackage{fullpage}
\usepackage{array}

\begin{document}
\renewcommand{\arraystretch}{1.75}


\section{Free Energy equations for different phases}

The model takes as input the free energy of all four phases assumed to be present in the simulation box. These phases are 

\begin{enumerate}
\item Fe metal containing dissolved sulfur
\item Pyrrhotite ($Fe_{1-\delta}S$)
\item Pyrite ($FeS_{2+\delta}$)
\item Environmental sulfur (i.e. sulfur as $H_2S$ in the gaseous or dissolved form)
\end{enumerate}

The free energies of the phases are given below in J\ mol$^{-1}$

\begin{equation}
  G_{\mathrm{metal}}(T,[S]) = 0
\label{g_met}
\end{equation}

\begin{equation}
  G_{\mathrm{pyrrhotite}}(T,\ \delta) = 250896 \ \delta^2 + 20.53T - 72050
\label{g_pht}
\end{equation}

\begin{equation}
  G_{\mathrm{pyrite}}(T,\ \delta) = 250896000 \ \delta^2 + 50.36T - 98710
\label{g_pyr}
\end{equation}

\begin{equation}
  G_{\mathrm{environment}}(T,[S]) = 0 + RT\ln[S]
\label{g_env}
\end{equation}


For the purpose of the phase field calculations, the free energy of the metal is taken to be zero and the other values are scaled accordingly. Additionally, sulfur is assumed to exist as a dilute solution in both the metal and the environment, therefore the chemical potential of sulfur in these phases (in J mol$^{-1}$) is given by

\begin{equation*}
  \mu_{S,i}(T) = RT\ln([S]); \quad \forall\ i \subset (\mathrm{metal,environment}) 
\end{equation*}

In the case of pyrrhotite and pyrite, the derivative of the molar free energy (equations \ref{g_pht} and \ref{g_pyr}) with sulfur content (or equivalently $\ \delta$ in each case) gives us the sulfur chemical potential (in J mol$^{-1}$) as shown below.

\begin{equation}
  \mu_{s,\mathrm{pyrrhotite}}(T) = 501792 \ \delta
\end{equation}

\begin{equation}
  \mu_{s,\mathrm{pyrite}}(T) = 501792000 \ \delta
\end{equation}


In this light, equations \ref{g_met}-\ref{g_env} can be rewritten in terms of the sulfur chemical potential, expressing the free energy of each phase in terms of local chemical potential of sulfur.



\begin{equation}
  G_{\mathrm{metal}}(T,\mu_S) = 0
\label{gmu_met}
\end{equation}

\begin{equation}
  G_{\mathrm{pyrrhotite}}(T,\mu_S) = 9.96\times10^{-7} \mu_S^2 + 20.53T - 72050
\label{gmu_pht}
\end{equation}

\begin{equation}
  G_{\mathrm{pyrite}}(T,\mu_S) = 9.96\times10^{-10} \mu_S^2 + 50.36T - 98710
\label{gmu_pyr}
\end{equation}

\begin{equation}
  G_{\mathrm{environment}}(T,\mu_S) = 0 + \mu_S
\label{gmu_env}
\end{equation}



\section{Phase field formulation}

The free energy of the system is specified by four phase fields, $\phi_{met}$,$\phi_{pht}$,$\phi_{env}$, and $\mu_{S}$ denoting respectively the phase fractions of the metal, pyrrhotite, the environment and the chemical potential of sulfur. In this formalism the grand potential of the system is given by

\begin{equation}
  \mathcal{F} = \sum\limits_{i<j}^{N,N} \sigma_{ij}(\phi_i\nabla\phi_j-\phi_j\nabla\phi_i)^2 + H_{ij}\phi_i^2\phi_j^2 + ((\phi_i^3+\phi_i^2\phi_j-\phi_j^2\phi_i-\phi_j^3)\times(w_i-w_j))
\label{eq:gr_pot}
\end{equation}

where $i$ represents different phases (in this case, pyrrhotite, pyrite, environment and metal). The overall expression for the grand potential of the system can then be decomposed into pairwise components as

\begin{equation}
  \mathcal{F} = \sum\limits_{i<j}^{N,N}\mathcal{F}_{ij}
\end{equation}


where $\mathcal{F}_{ij}$ is the expression inside the summation in equation \ref{eq:gr_pot}. Specifically, in our system

\begin{equation}\label{eq:split_f}
  \begin{split}
  \mathcal{F} &= \mathcal{F}_{\mathrm{pyrrhotite,metal}} + \mathcal{F}_{\mathrm{pyrrhotite,environment}} + \mathcal{F}_{\mathrm{pyrrhotite,pyrite}} \\
              &+ \mathcal{F}_{\mathrm{environment,metal}} + \mathcal{F}_{\mathrm{environment,pyrite}} \\
              &+ \mathcal{F}_{\mathrm{metal,pyrite}}    
  \end{split}
\end{equation}


The equation for the evolution of each phase field is given by


\begin{equation}\label{eq:pf_evol}
  \dot{\phi_i} = \sum\limits_{i\neq j}^N -M_{ij}\left(  \frac{\partial\mathcal{F}_{ij}}{\partial \phi_i} - \frac{\partial\mathcal{F}_{ij}}{\partial \phi_j}  \right)
\end{equation}


Plugging equation \ref{eq:gr_pot} back into equation \ref{eq:pf_evol}, we get

\begin{equation}
  \dot{\phi_i} = \sum\limits_{i\neq j}^N -M_{ij}\left[ -\sigma_{ij}\left(\phi_i\nabla^2\phi_j-\phi_j\nabla^2\phi_i\right) + 2H_{ij}\ \phi_i\phi_j (\phi_j-\phi_i) + 2(w_i-w_j)(\phi_i+\phi_j)^2\right]
\end{equation}













\section{Physical parameters from external sources}

\subsection{Diffusivities}


\begin{center}
    \begin{tabular}{ | c | p{5cm} | p{5cm} | c |}
    \hline
    Parameter & Description & Value (m$^2$/s) & Source \\
    \hline
    $D_{Fe}^{Fe}$ & Self diffusivity of BCC Fe & $7.87 \times 10^{-7} \  e^{\left(\frac{-57900}{R T}\right)}$ & \cite{MDBFMM09} \\
    \hline
    $D_{S}^{Fe}$ & Diffusivity of sulfur in BCC Fe & $ 1.78 \ e^{\left(\frac{-205000}{R T}\right)}$ & \cite{DOCACF82} \\

    \hline
    $D_{Fe}^{mkw}$ & Diffusivity of iron in mackinawite & $ 3.37 \times 10^{-16} \ e^{\left(\frac{-15500}{R T}\right)}$ & \cite{SOCCJA11} \\

    \hline
    $D_{Fe}^{pht}$ & Diffusivity of iron in pyrrhotite & $ 10^{\left(\frac{-7056}{T} - 2.679 \right)}$ & \cite{MDPFFH15} \\

    \hline
    $D_{S}^{pyr}$ & Diffusivity of sulfur in pyrite & $ 1.75 \times 10^{-14} \ e^{\left(\frac{-132100}{R T}\right)}$ & \cite{ROOSEW09} \\

    \hline
    $D_{Fe}^{pyr}$ & Diffusivity of iron in pyrite & $ 2.5 \times 10^{-16} \ e^{\left(\frac{-41840}{R T}\right)}$ & \cite{CSAPJC75} \\

    \hline
    $D_{S}^{env}$ & Diffusivity of sulfur in water & $ 1.73 \times 10^{-9}$ & \cite{DC3KAT94} \\

    \hline
    $D_{Fe}^{env}$ & Diffusivity of iron in water & $ 7.19 \times 10^{-10}$ & \cite{MOPMJD02} \\

    \hline

    \end{tabular}
\end{center}



\subsection{Dissolution rates}


\begin{center}
    \begin{tabular}{ | c | p{8cm} | c |}
    \hline
    Material & Dissolution rate (in nm/s) & Source \\

    \hline
    Fe & $0.257 [OH^-]^{0.6} e^{\frac{82012(V+0.45)}{R T}}$ & \cite{EKASRC72} \\

    \hline
    Mackinawite & $0.015$ & \cite{CMESLS08} \\

    \hline
    Pyrrhotite & $289.15 e^{\frac{-65900}{R T}} \times 10^{-1.46 pH}$ & \cite{PDAMPC14} \\

    \hline
    Pyrite & $0.00017244$ & \cite{ISACMP08} \\

    \hline

    \end{tabular}
\end{center}






\subsection{Sulfidation rates}


\begin{center}
    \begin{tabular}{ | c | c | p{8cm} | c |}
    \hline
    Material & Environment & Sulfidation rate (in nm/s) & Source \\

    \hline
    Fe & Gas & $ 10^{0.00473 T - 5.645 + \frac{\mu_S+63562}{R T}}   $ & \cite{ACPPRJ04} \\

    \hline
    Pyrrhotite & Gas & $e^{\frac{-11766}{T} - 0.6478}$ & \cite{MGISFH15} \\

    \hline
    Pyrite & Gas & $7.45 \times 10^8 \ e^{\frac{-98400}{R T}}$ & \cite{KOGSRP03} \\

    \hline
    Pyrrhotite & Liquid & $0.01372 \times 10^{-9} + 0.04356 \times 10^{-9} e^{\frac{\mu_S}{R T}}$ & \cite{COAETR90} \\

    \hline

    \end{tabular}
\end{center}





\bibliographystyle{plain}
\bibliography{references}




\end{document}
